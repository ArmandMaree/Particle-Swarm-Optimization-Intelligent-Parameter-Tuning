%% bare_conf.tex
%% V1.4b
%% 2015/08/26
%% by Michael Shell
%% See:
%% http://www.michaelshell.org/
%% for current contact information.
%%
%% This is a skeleton file demonstrating the use of IEEEtran.cls
%% (requires IEEEtran.cls version 1.8b or later) with an IEEE
%% conference paper.
%%
%% Support sites:
%% http://www.michaelshell.org/tex/ieeetran/
%% http://www.ctan.org/pkg/ieeetran
%% and
%% http://www.ieee.org/

%%*************************************************************************
%% Legal Notice:
%% This code is offered as-is without any warranty either expressed or
%% implied; without even the implied warranty of MERCHANTABILITY or
%% FITNESS FOR A PARTICULAR PURPOSE! 
%% User assumes all risk.
%% In no event shall the IEEE or any contributor to this code be liable for
%% any damages or losses, including, but not limited to, incidental,
%% consequential, or any other damages, resulting from the use or misuse
%% of any information contained here.
%%
%% All comments are the opinions of their respective authors and are not
%% necessarily endorsed by the IEEE.
%%
%% This work is distributed under the LaTeX Project Public License (LPPL)
%% ( http://www.latex-project.org/ ) version 1.3, and may be freely used,
%% distributed and modified. A copy of the LPPL, version 1.3, is included
%% in the base LaTeX documentation of all distributions of LaTeX released
%% 2003/12/01 or later.
%% Retain all contribution notices and credits.
%% ** Modified files should be clearly indicated as such, including  **
%% ** renaming them and changing author support contact information. **
%%*************************************************************************


% *** Authors should verify (and, if needed, correct) their LaTeX system  ***
% *** with the testflow diagnostic prior to trusting their LaTeX platform ***
% *** with production work. The IEEE's font choices and paper sizes can   ***
% *** trigger bugs that do not appear when using other class files.       ***                          ***
% The testflow support page is at:
% http://www.michaelshell.org/tex/testflow/

\documentclass[conference]{IEEEtran}

% *** MATH PACKAGES ***
%
\usepackage{amssymb}


\hyphenation{op-tical net-works semi-conduc-tor}


\begin{document}
\title{Particle Swarm Optimization: Intelligent Parameter Tuning}
\author{\IEEEauthorblockN{Armand Maree}
\IEEEauthorblockA{Department of Computer Science\\
University of Pretoria\\}}
\maketitle

\begin{abstract}
Particle Swarm Optimization (PSO) is a optimization algorithm that models the flocking behaviour of certain species of animals in order to solve stochastic real-value problems. These algorithms use control parameters to make slight adjustments to how the swarm behaves. In this paper the author will be discussing the results of using a brute force approach to find these control parameters.
\end{abstract}

\IEEEpeerreviewmaketitle



\section{Introduction}
% no \IEEEPARstart
A Particle Swarm Optimisor (PSO) is, as described by Lazincia \cite{lazinica2009particle}, an optimization algorithm that attempts to emulate the flocking/schooling behaviour of birds or fish when searching for food. This optimization can occur in a number of dimensions, each spanning across $\mathbb{R}$ \cite{kennedy1997discrete}. Each particle in the swarm has two primary components of influence: cognative and social. A third component, called the inertia weight, is used to regulate the maximum velocity that a particle can move at in a certain direction \cite{eberhart2000comparing}. In this paper, the author will explore the feasibility of finding the optimum value for each of these components by using a brute force search in a subset.

\section{Background}



\section{Implementation}



\section{Research Results}



\section{Conclusion}


\bibliographystyle{unsrt}
\bibliography{bibliography}

\end{document}


